% Compile with:
% latexmk -pdf -pvc -interaction=nonstopmode
%\documentclass[aspectratio=169,10pt,draft]{beamer}
\documentclass[aspectratio=169,10pt]{beamer}
\usetheme{UniBern}
\title{Adaptation Mechanisms Of Zebrafish Respiratory Organ To Endurance Training}
\author{David Haberthür \and
	Dea Aaldijk \and
	Matthias Messerli \and
	Fluri A.\ M.\ Wieland \and
	Oleksiy Khoma \and
	Helena Röss \and
	Ruslan Hlushchuk}
\institute{Institute of Anatomy\\University of Bern\\Switzerland}
\date{June 5, 2019 | \href{https://www.bruker.com/events/micro-ct-users-meeting.html}{Bruker micro-CT Users Meeting 2019}}

%\includeonlyframes{current}
%then....
%\begin{frame}[label=current]
%\end{frame}

\usepackage[english]{babel}
\usepackage{microtype}
\usepackage[backend=biber, style=numeric, url=false, isbn=false, maxbibnames=1, sorting=none]{biblatex}
	\addbibresource{../../Documents/library.bib}
\usepackage{graphicx}
\usepackage[detect-all=true, range-phrase=--, range-units=single, binary-units=true]{siunitx}
\usepackage[absolute,overlay]{textpos} %for the \source{} command
\usepackage{gitinfo2}
\usepackage[version=4]{mhchem}
\usepackage{xspace}
\usepackage{ccicons}

% Some often used abbreviations
\newcommand{\imsize}{\linewidth} % globally set image width
\newcommand{\everyframe}{1} % use only every nth frame for the movies
\newlength\imagewidth % needed for scalebars
\newlength\imagescale % needed for scalebars
\newcommand{\uct}{\si{\micro}CT\xspace} % make our life easier

% Easily fill a frame with the whole image.
% Based on http://tex.stackexchange.com/a/334758/828, http://tex.stackexchange.com/a/244103/828 and ubTitleHeight and ubFooterHeight found in the Unibe Beamer template.
\newcommand{\fullframeimage}[1]{%
	\begin{tikzpicture}[remember picture,overlay]%
		%\node[xshift=0,yshift=-0.085\paperheight-0.016\paperheight/2] at (current page.center){\includegraphics[width=\paperwidth]{#1}};%
		\node[xshift=0,yshift=0] at (current page.center){\includegraphics[width=\paperwidth]{#1}};%		
	\end{tikzpicture}%
}

% Acknowledge things in the lower right of the slide
% Based on http://tex.stackexchange.com/a/48485/828
\newcommand{\source}[1]{%
	\begin{textblock*}{4cm}(8cm,8cm)%
		\begin{beamercolorbox}[ht=0.5cm,right]{framesource}%
			\tiny\usebeamerfont{framesource}\usebeamercolor[fg]{framesource} {#1}%
		\end{beamercolorbox}%
	\end{textblock*}%
}

% Define us our custom footer
\defbeamertemplate{footline}{unibe}{%
	\hspace*{0.6cm}%
	v. \href{https://github.com/habi/20190605_BrukerUserMeeting/commit/\gitHash}{\gitAbbrevHash}%
	\hspace*{\fill}%
	\insertframenumber\,/\,\inserttotalframenumber%
	\hspace*{1.2cm}%
	\vskip2pt%
}
\setbeamertemplate{footline}[unibe]

% Format bibliography for beamer
% http://tex.stackexchange.com/a/10686/828
\renewbibmacro{in:}{}
% http://tex.stackexchange.com/a/13076/828
\AtEveryBibitem{\clearfield{journaltitle}}
\AtEveryBibitem{\clearfield{pages}}
\AtEveryBibitem{\clearfield{volume}}
\AtEveryBibitem{\clearfield{number}}
\AtEveryBibitem{\clearfield{editors}}

\begin{document}
% We want no footline on the title page, http://tex.stackexchange.com/a/18829/828 helps
{%
	\setbeamertemplate{footline}{}%
	\begin{frame}%
		\maketitle
	\end{frame}%
}

\begin{frame}
	\frametitle{Contents}
	\tableofcontents
\end{frame}

\begin{frame}
	\frametitle{Hello!}
	\begin{itemize}
		\item<1-> University of Bern, Switzerland
		\item<1-> Institute of Anatomy
		\item<1-> Group for topographic and clinical Anatomy
		\item<1-> \uct-Team: Ruslan Hlushchuk, David Haberthür, Oleksiy-Zakhar Khoma, Fluri Wieland, Carlos Correa Shokiche
		\item<1-> Biomedical research
		\begin{itemize}
			\item microangioCT~\cite{Hlushchuk2018}: Tumor vasculature, angiogenesis in the heart, musculature
			\item Cancer research: Melanoma
			\item Lung imaging: Tumor detection and classification
			\item Physiology: Zebrafish musculature and gills
		\end{itemize}
		\item<1-> SkyScan 1172 \& 1272 \visible<2->{\& 2214}
	\end{itemize}
\end{frame}

\begin{frame}
	\frametitle{Why?}
	\begin{itemize}
		\item Adaptation of respiratory organ to endurance training
		\item Study \ce{O2} metabolism
	\end{itemize}
\end{frame}

\begin{frame}
	\frametitle{Oxygen pathway}
	\begin{columns}
		\begin{column}[T]{0.6\linewidth}
			\begin{itemize}
				\item<1-> Insects: Spiracles \& Trachea
				\item<2-> Bony fishes: Gills
				\item<3-> Mammals: Lungs
			\end{itemize}
		\end{column}
		\begin{column}[T]{0.4\linewidth}
			\only<1>{\includegraphics[width=\imsize]{./img/2746294525_52566921c8_o.jpg}\source{flic.kr/p/5bFtFB | \ccbyncsa}}
			\only<2>{\includegraphics[width=\imsize]{./img/2267404278_88e0f1f8b6_o.jpg}\source{flic.kr/p/4sn3id | \ccbyncsa}}
			\only<3>{\includegraphics[width=\imsize]{./img/Anim1754_-_Flickr_-_NOAA_Photo_Library.jpg}\source{enwp.org/bluewhale | \ccPublicDomain}}
		\end{column}
	\end{columns}
\end{frame}

\begin{frame}
	\frametitle{How?}
	\begin{itemize}
		\item Fish training
		\item Scanning
		\item Delination
		\item Analysis
	\end{itemize}
\end{frame}

\begin{frame}
	\frametitle{What?}
	\begin{itemize}
		\item Gill volume
		\item Gill complexity
	\end{itemize}
\end{frame}

\begin{frame}
	\frametitle{Wee!}
	\begin{itemize}
		\item We conclude that the zebrafish respiratory organ has a high plasticity, and after endurance training increases its volume and changes its structure in order to facilitate \ce{O2} uptake.
	\end{itemize}
\end{frame}

\begin{frame}
	\frametitle{Thanks!}
	\begin{columns}
		\begin{column}{0.58\linewidth}
	\begin{itemize}
		\item<1-> Topographic and clinical Anatomy
		\begin{itemize}
			\item<1-> Dea Aaldijk, Matthias Messerli
			\item<1-> Ruslan Hlushchuk, Valentin Djonov
			\item<1-> Fluri A.\ M.\ Wieland, Oleksiy Khoma
			\item<1-> Sarya Fark, Helena Röss
		\end{itemize}
		\item<1-> People@ana.unibe.ch
		\begin{itemize}
			\item<1-> Werner Graber, Jeannine Wagner-Zimmermann, Beat Haenni
			\item<1-> Regula Buergy, Eveline Yao and Sara Soltermann
			\item<1-> Marcos Sande, Carolina Garcia
			\item<1-> Ines Marquez, Xavier Langa and Alexander Uwe Ernst
		\end{itemize}
		\item<1-> SNF
		\item<2-> You, for listening
		\item<3-> Questions?
	\end{itemize}
		\end{column}
		\begin{column}{0.38\linewidth}
			\includegraphics<1->[width=\linewidth]{./img/team}
		\end{column}	
	\end{columns}	
\end{frame}

\begin{frame}
	\frametitle{References}
	%\renewcommand*{\bibfont}{\tiny}
	\setbeamertemplate{bibliography item}{\insertbiblabel}
	\printbibliography
\end{frame}

\begin{frame}
	\frametitle{Colophon}
	\begin{itemize}
		\item \href{http://intern.unibe.ch/dienstleistungen/corporate_design_und_vorlagen/praesentationen/index_ger.html}{Template from Corporate Design und Vorlagen, University of Bern}
		\item \href{https://github.com/habi/20190605_BrukerUserMeeting}{Full (\LaTeX) source code of this \textsc{beamer} presentation is available on GitHub}
	\end{itemize}
\end{frame}

\end{document}
\usepackage{gitinfo2}
